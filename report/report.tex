\section{如果没有一级标题,那么这就是一级标题}

\subsection{这是二级标题}

\subsubsection{这是三级标题}

\paragraph{六级标题将会变成paragraph}

\par 这是一段正常的文字,\textbf{这里是粗体},\textit{这里是斜体}。

\par 可以使用Latex的相关行内公式,如$a^2+b^2=c^2$。

\par 可以使用行内代码,但是需要注意的是行内代码不区分代码语言,如\mintinline{text}|print("Hello World")|。

\par 可以使用超链接功能,如\href{https://www.cleversmall.com}{Cleversmall}。

\par 可以使用文献引用功能,但必须配合\textbf{bibtex}使用,该功能在markdown中只起到一个标志的作用,转为latex后才会生效,如文献\cite{name2024paper}

\par 可以使用图片插入和图片引用跳转功能,如图\ref{fig1}所示。

\begin{figure}[ht]
    \centering
    \includegraphics[width=\textwidth]{./figure/latex_bird.png}
    \caption{Latex Bird}
    \label{fig1}
\end{figure}

\par 可以插入表格,并识别对齐方式,并使用引用的跳转,如表\ref{tab1}所示。

\begin{table}[ht]
    \centering
    \caption{测试表格}
    \begin{tabular}{rcl}
        \toprule
        \textbf{标题1}&\textbf{这里居中对齐}&\textbf{标题2} \\
        \midrule
          右对齐了  &     内容2      &  左对齐了   \\
                    &     内容3      &             \\
        \bottomrule
    \end{tabular}
    \label{tab1}
\end{table}

\par 可以使用无序列表功能,如下。

\begin{itemize}
    \item 一个项目
    \item 两个项目
    \item 三个项目

\end{itemize}
\par 也可以使用有序列表功能,如下。

\begin{enumerate}
    \item 内容一
    \item 内容二
    \item 内容三

\end{enumerate}
\par 可以使用行间公式功能,但请使用标准latex公式格式书写其中内容。
\begin{equation}
\begin{bmatrix}
a & b \\ c & d
\end{bmatrix}
\neq
\begin{matrix}
e & f \\ g & h
\end{matrix}
\end{equation}
\par 可以使用行间代码功能,并识别语言。

\begin{minted}{python}
import numpy

print("Hello World")
\end{minted}

\par 还有好多功能理论上,后面慢慢加吧


